%%%%%%%%%%%%%%%%%%%%%%%%%%%%%%%%%%%%%%%%%%%%%%%%%%%%%%%%%%%%%%%%%%%%%%%%%%%%%%%%%%%%%%%%%%%%%%%%%%%%%%%%%%%
%                                           PACKAGES                                                      %
%%%%%%%%%%%%%%%%%%%%%%%%%%%%%%%%%%%%%%%%%%%%%%%%%%%%%%%%%%%%%%%%%%%%%%%%%%%%%%%%%%%%%%%%%%%%%%%%%%%%%%%%%%%
\documentclass[12pt, fleqn]{article}
\usepackage{amsmath, amsfonts, amsthm, amssymb, graphicx, enumitem, mathtools, MnSymbol, relsize, cancel}
\usepackage{siunitx}
\DeclareSIUnit\angstrom{\text{\AA}}
\usepackage{pdfpages}
\usepackage{graphicx}
\usepackage[utf8]{inputenc}
\usepackage{biblatex}
\usepackage{pythontex}
\usepackage{listings}
\usepackage[pdftex,pdfpagelabels,bookmarks,hyperindex,hyperfigures]{hyperref}
\hypersetup{colorlinks=true,allcolors=blue}
\usepackage{hypcap}
\usepackage{float}
\usepackage{geometry}
\geometry{margin=1in}
%%%%%%%%%%%%%%%%%%%%%%%%%%%%%%%%%%%%%%%%%%%%%%%%%%%%%%%%%%%%%%%%%%%%%%%%%%%%%%%%%%%%%%%%%%%%%%%%%%%%%%%%%%%
%                                           REFERENCE FILE                                                %
%%%%%%%%%%%%%%%%%%%%%%%%%%%%%%%%%%%%%%%%%%%%%%%%%%%%%%%%%%%%%%%%%%%%%%%%%%%%%%%%%%%%%%%%%%%%%%%%%%%%%%%%%%%
\usepackage[export]{adjustbox}
\graphicspath{{images/}}
%%%%%%%%%%%%%%%%%%%%%%%%%%%%%%%%%%%%%%%%%%%%%%%%%%%%%%%%%%%%%%%%%%%%%%%%%%%%%%%%%%%%%%%%%%%%%%%%%%%%%%%%%%%
%                                          PREPARE TITLE AND ABSTRACT                                     %
%%%%%%%%%%%%%%%%%%%%%%%%%%%%%%%%%%%%%%%%%%%%%%%%%%%%%%%%%%%%%%%%%%%%%%%%%%%%%%%%%%%%%%%%%%%%%%%%%%%%%%%%%%%
\title {
    \normalsize{UC Berkeley}\\
    \large{{EE140: Analog Integrated Circuit Devices\\Fall 2022\\Professor Ricky Muller\\}}
    \vspace{0.5ex}
    \Huge{Final Project}
    \vspace{0.5ex}
}
\addbibresource{references.bib}
\author{Tarik Fawal}
\date{5 December 2022}
\usepackage{array}
\newcolumntype{C}[1]{>{\centering\arraybackslash}m{#1}}
\newcolumntype{N}{@{}m{0pt}@{}}
\begin{document}
%%%%%%%%%%%%%%%%%%%%%%%%%%%%%%%%%%%%%%%%%%%%%%%%%%%%%%%%%%%%%%%%%%%%%%%%%%%%%%%%%%%%%%%%%%%%%%%%%%%%%%%%%%%
%                                           GENERATE TITLE                                                %
%%%%%%%%%%%%%%%%%%%%%%%%%%%%%%%%%%%%%%%%%%%%%%%%%%%%%%%%%%%%%%%%%%%%%%%%%%%%%%%%%%%%%%%%%%%%%%%%%%%%%%%%%%%
\maketitle
\tableofcontents
\flushbottom
    \section*{Preface}
        \textit{\emph{This project report was created using \LaTeX.  The answers to questions were obtained through the course website, notes, textbook, and lecture videos.  I pledge that I have not plagiarized my solutions in any way, and the work presented here is my own.}}
%%%%%%%%%%%%%%%%%%%%%%%%%%%%%%%%%%%%%%%%%%%%%%%%%%%%%%%%%%%%%%%%%%%%%%%%%%%%%%%%%%%%%%%%%%%%%%%%%%%%%%%%%%%
%                                              OVERVIEW                                                   %
%%%%%%%%%%%%%%%%%%%%%%%%%%%%%%%%%%%%%%%%%%%%%%%%%%%%%%%%%%%%%%%%%%%%%%%%%%%%%%%%%%%%%%%%%%%%%%%%%%%%%%%%%%%
\newpage
\section{Overview}
\subsection{Design Specifications}
    The amplifier was designed according to the specifications listed below.\\[0.5cm]
    \begin{table}[H]
        \centering
        \setlength{\tabcolsep}{4pt}
        \renewcommand{\arraystretch}{2}
        \begin{tabular}{|l|c|}
            \hline
            \textbf{Technology} & Gpdk045\\
            \hline
            \textbf{Power Supplies} & GND ($0V$), VDDL ($< 1.1\,V$), VDDH ($< 1.8\,V$)\\
            \hline
            \textbf{Closed Loop DC Gain} & $2$\\
            \hline
            \textbf{Load} & $800\,\Omega$, $30\,pF$ (output at capacitor)\\
            \hline
            \textbf{Settling Time = $max\{{T_{settle}}_{rise}, {T_{settle}}_{fall}\}$\footnote{Settling time is defined as the maximum of the rising and falling settling times for a $1.4\,V$ output step.}} & To be calculated for $1.4\,V$ output step rising and falling\\
            \hline
            \textbf{Total Error} & $\leq 0.2\,\%$\\
            \hline
            \textbf{Power Consumption} & $\leq 0.75\,mW$\\
            \hline
            \textbf{Output Voltage Swing} & $\geq 1.4\,V$\\
            \hline
            \textbf{Maximum Current Mirror Ratio} & $20$\\
            \hline
            \textbf{Maximum Total Capacitance\footnote{Defined as total capacitance of explicit capacitors added to the design. This excludes the load and the feedback capacitors.}} & $2\;pF$\\
            \hline
            \textbf{Maximum Total Resistance} & $100\,M\Omega$\\
            \hline
            \textbf{CMRR at DC} & $\geq 65\,dB$\\
            \hline
            \textbf{PSRR at DC} & $\geq 50\,dB$\\
            \hline
            \textbf{Phase Margin} & $\geq {45}^\circ$\\
            \hline
            \textbf{Figure of Merit (FoM) to Maximize} & $\mathlarger{\frac{{10}^{-9}}{T_{settle} \times P_{total}}}$\\
            \hline
        \end{tabular}
        \caption{Table of required specifications the amplifier is expected to reach.
        \label{tab:specs1}} 
    \end{table}
\subsection{Specifications of Finished Amplifier}
    The resulting specifications of the final amplifier are listed below.\\[0.5cm]
    \begin{table}[H]
        \centering
        \setlength{\tabcolsep}{4pt}
        \renewcommand{\arraystretch}{2}
        \begin{tabular}{|l|c|}
            \hline
            \textbf{Technology} & Gpdk045\\
            \hline
            \textbf{Power Supplies} & GND ($0V$), VDDL ($< 1.1\,V$), VDDH ($< 1.8\,V$)\\
            \hline
            \textbf{Closed Loop DC Gain} & $2$\\
            \hline
            \textbf{Load} & $800\,\Omega$, $30\,pF$ (output at capacitor)\\
            \hline
            \textbf{Settling Time = $max\{{T_{settle}}_{rise}, {T_{settle}}_{fall}\}$\footnote{Settling time is defined as the maximum of the rising and falling settling times for a $1.4\,V$ output step.}} & To be calculated for $1.4\,V$ output step rising and falling\\
            \hline
            \textbf{Total Error} & $\leq 0.2\,\%$\\
            \hline
            \textbf{Power Consumption} & $\leq 0.75\,mW$\\
            \hline
            \textbf{Output Voltage Swing} & $\geq 1.4\,V$\\
            \hline
            \textbf{Maximum Current Mirror Ratio} & $20$\\
            \hline
            \textbf{Maximum Total Capacitance\footnote{Defined as total capacitance of explicit capacitors added to the design. This excludes the load and the feedback capacitors.}} & $2\;pF$\\
            \hline
            \textbf{Maximum Total Resistance} & $100\,M\Omega$\\
            \hline
            \textbf{CMRR at DC} & $\geq 65\,dB$\\
            \hline
            \textbf{PSRR at DC} & $\geq 50\,dB$\\
            \hline
            \textbf{Phase Margin} & $\geq {45}^\circ$\\
            \hline
            \textbf{Figure of Merit (FoM) to Maximize} & $\mathlarger{\frac{{10}^{-9}}{T_{settle} \times P_{total}}}$\\
            \hline
        \end{tabular}
        \caption{Table of results.
        \label{tab:specs2}} 
    \end{table}
%%%%%%%%%%%%%%%%%%%%%%%%%%%%%%%%%%%%%%%%%%%%%%%%%%%%%%
%                     SCHEMATIC                      %
%%%%%%%%%%%%%%%%%%%%%%%%%%%%%%%%%%%%%%%%%%%%%%%%%%%%%%
\newpage
\subsection{Schematic of Amplifier}
%%%%%%%%%%%%%%%%%%%%%%%%%%%%%%%%%%%%%%%%%%%%%%%%%%%%%%
%                     OPERATION                      %
%%%%%%%%%%%%%%%%%%%%%%%%%%%%%%%%%%%%%%%%%%%%%%%%%%%%%%
\newpage
\subsection{Description of Circuit Operation}
%%%%%%%%%%%%%%%%%%%%%%%%%%%%%%%%%%%%%%%%%%%%%%%%%%%%%%%%%%%%%%%%%%%%%%%%%%%%%%%%%%%%%%%%%%%%%%%%%%%%%%%%%%%
%                                               DESIGN                                                    %
%%%%%%%%%%%%%%%%%%%%%%%%%%%%%%%%%%%%%%%%%%%%%%%%%%%%%%%%%%%%%%%%%%%%%%%%%%%%%%%%%%%%%%%%%%%%%%%%%%%%%%%%%%%
\newpage
\section{Design}
%%%%%%%%%%%%%%%%%%%%%%%%%%%%%%%%%%%%%%%%%%%%%%%%%%%%%%
%                      APPROACH                      %
%%%%%%%%%%%%%%%%%%%%%%%%%%%%%%%%%%%%%%%%%%%%%%%%%%%%%%
\subsection{Design Approach}
% add some of the derived equations here as a prologue
I decided to begin with a basic 2-stage CMOS amplifier.  My initial schematic is below.\\[0.25cm]
% \includegraphics[scale=0.15, center]{null.PNG}
I wanted to start by sizing the transistors based on the intrinsic gain of each stage and the $g_m / I_D$ using the LUT's from Matlab.  I decided that a minimum $g_m / I_D$ of 10 would be a good starting point, and an intrinsic gain that facilitated using a reasonable channel length.  The static gain equation above has the intrinsic gain of the amplifier as its only unknown, so I used that equation, and split the static gain and dynamic gain error evenly, because the also seemed like a good starting point.
%%%%%%%%%%%%%%%%%%%%%%%%%%%%%%%%%%%%%%%%%%%%%%%%%%%%%%
%                      EQUATIONS                     %
%%%%%%%%%%%%%%%%%%%%%%%%%%%%%%%%%%%%%%%%%%%%%%%%%%%%%%
\newpage
\subsection{Design Equations}
%%%%%%%%%%%%%%%%%%%%%%%%%%%%%%%%%%%%%%%%%%%%%%%%%%%%%%
%                    METHODOLOGY                     %
%%%%%%%%%%%%%%%%%%%%%%%%%%%%%%%%%%%%%%%%%%%%%%%%%%%%%%
\newpage
\subsection{Design Methodology}
%%%%%%%%%%%%%%%%%%%%%%%%%%%%%%%%%%%%%%%%%%%%%%%%%%%%%%%%%%%%%%%%%%%%%%%%%%%%%%%%%%%%%%%%%%%%%%%%%%%%%%%%%%%
%                                 TRANSISTOR / BIAS SUMMARY                                               %
%%%%%%%%%%%%%%%%%%%%%%%%%%%%%%%%%%%%%%%%%%%%%%%%%%%%%%%%%%%%%%%%%%%%%%%%%%%%%%%%%%%%%%%%%%%%%%%%%%%%%%%%%%%
\newpage
\section{Transistor and Bias Summary}
%%%%%%%%%%%%%%%%%%%%%%%%%%%%%%%%%%%%%%%%%%%%%%%%%%%%%%
%                SPECIFICATIONS                      %
%%%%%%%%%%%%%%%%%%%%%%%%%%%%%%%%%%%%%%%%%%%%%%%%%%%%%%
\subsection{Table of Specifications}
%%%%%%%%%%%%%%%%%%%%%%%%%%%%%%%%%%%%%%%%%%%%%%%%%%%%%%%%%%%%%%%%%%%%%%%%%%%%%%%%%%%%%%%%%%%%%%%%%%%%%%%%%%%
%                                        DISCUSSION                                                       %
%%%%%%%%%%%%%%%%%%%%%%%%%%%%%%%%%%%%%%%%%%%%%%%%%%%%%%%%%%%%%%%%%%%%%%%%%%%%%%%%%%%%%%%%%%%%%%%%%%%%%%%%%%%
\newpage
\section{Discussion}
%%%%%%%%%%%%%%%%%%%%%%%%%%%%%%%%%%%%%%%%%%%%%%%%%%%%%%%%%%%%%%%%%%%%%%%%%%%%%%%%%%%%%%%%%%%%%%%%%%%%%%%%%%%
%                                        CONCLUSION                                                       %
%%%%%%%%%%%%%%%%%%%%%%%%%%%%%%%%%%%%%%%%%%%%%%%%%%%%%%%%%%%%%%%%%%%%%%%%%%%%%%%%%%%%%%%%%%%%%%%%%%%%%%%%%%%
\newpage
\section{Conclusion}
%%%%%%%%%%%%%%%%%%%%%%%%%%%%%%%%%%%%%%%%%%%%%%%%%%%%%%%%%%%%%%%%%%%%%%%%%%%%%%%%%%%%%%%%%%%%%%%%%%%%%%%%%%%
%                                           END OF DOCUMENT                                               %
%%%%%%%%%%%%%%%%%%%%%%%%%%%%%%%%%%%%%%%%%%%%%%%%%%%%%%%%%%%%%%%%%%%%%%%%%%%%%%%%%%%%%%%%%%%%%%%%%%%%%%%%%%%
\end{document}
